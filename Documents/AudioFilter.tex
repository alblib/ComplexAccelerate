\documentclass{article}
\usepackage{amsmath, amssymb, amsthm}
\usepackage{xurl, hyperref}
\title{Audio Filters}
\author{Woohyun \textsc{Rim}}
\date{\today}
\begin{document}
\maketitle
\tableofcontents
\section{Analog Filter Units}
\subsection{First-order Filters}
\subsubsection{Shelving Filters}
First-order shelving filters can be constructed by simply adding a bottom constant function and a low- or high-pass filter. \cite{first-order-shelving} 
\begin{align}
H(s) = A + B \frac{1}{s+\omega_0} \rightarrow C\frac{s+a}{s+b},
\end{align}
where $a$, $b$, and $C$ are positive.
We want to express this transfer function into these three variables:
\begin{itemize}
\item gain at low frequency,
\item gain at high frequency, and
\item the center frequency (where the gain is the geometric average in amplitude sense or the arithmetic average in decibel sense of the lowest and the highest gains).
\end{itemize}
\par
The gain at the low frequency is,
\begin{align}
\left.H(s)\right|_{s=2\pi i f, f\rightarrow 0} = \left.H(s)\right|_{s=0} = C\,\frac{a}{b}.
\end{align}
\par
The gain at the low frequency is,
\begin{align}
\left.H(s)\right|_{s=2\pi i f, f\rightarrow \infty} = C.
\end{align}
\par
The center frequency $f_c$ satisfies
\begin{align}
\left|H(s=2\pi i f_c)\right| = C\sqrt{\frac{a}{b}}.
\end{align}
The left side expands:
\begin{align}
\left|H(s=2\pi i f_c)\right| =C \sqrt{\frac{a^2+(2\pi f_c)^2}{b^2+(2\pi f_c)^2}}.
\end{align}
Thus, the condition for $f_c$ simplifies:
\begin{align}
&\sqrt{\frac{a}{b}}
 = 
 \sqrt{\frac{a^2+(2\pi f_c)^2}{b^2+(2\pi f_c)^2}}
\\ \Rightarrow\quad&
a(b^2+(2\pi f_c)^2) = b(a^2+(2\pi f_c)^2)
\\ \Rightarrow\quad&
(b-a)(2\pi f_c)^2 + ab(a-b) = 0
\\ \stackrel{\text{unless $a=b$}}{\Rightarrow}\quad&
(2\pi f_c)^2 = ab
\end{align}

\par\par
To find the expression of the transfer function in terms of $H(0)$, $H(\infty)$, and $f_c$,
let us find the expression of $a$, $b$, and $C$ in terms of those variables.
\begin{align}
&\frac{a}{b} = \frac{H(0)}{H(\infty)},\quad (2\pi f_c)^2 = ab
\\ \Rightarrow\quad&
a = (2\pi f_c)\sqrt{\frac{H(0)}{H(\infty)}},
\quad
b = (2\pi f_c)\sqrt{\frac{H(\infty)}{H(0)}};
\\
& C = H(\infty).
\end{align}

\par
Thus, the transfer function in terms of meaningful parameters is written by
\begin{align}
H(s) = C\,\frac{s+a}{s+b} &= H(\infty) \frac{\frac{s}{2\pi f_c}+\sqrt{\frac{H(0)}{H(\infty)}}}{\frac{s}{2\pi f_c}+\sqrt{\frac{H(\infty)}{H(0)}}}
\\&=\dfrac{s\sqrt{H(\infty)}+\omega_c\sqrt{H(0)}}{\dfrac{s}{\sqrt{H(\infty)}}+\dfrac{\omega_c}{\sqrt{H(0)}}},
\end{align}
where $\omega_c = 2\pi f_c$.

\subsection{Second-order Filters}
\subsubsection{Shelving Filters}
The general second-order filter may be represented by
\begin{align}
H(s) = \frac{a s^2 + b s + c}{d s^2 + e s+ f}.
\end{align}
Here we will assume that the plot in log-log scale is rotational-symmetric.

\begin{thebibliography}{100}
\bibitem{first-order-shelving} \textit{Low and High Shelf Filters,} \url{https://ccrma.stanford.edu/~jos/fp/Low_High_Shelf_Filters.html}
\bibitem{second-order-shelving} \textit{Shelving filter,} \url{https://www.recordingblogs.com/wiki/shelving-filter}
\bibitem{analog-filter-design} Jonathan S. Abel and David P. Berners, \textit{Filter Design Using Second-Order Peaking and Shelving Sections,} \textit{Proceedings ICMC 2004} (2004) \url{http://hdl.handle.net/2027/spo.bbp2372.2004.152}
\end{thebibliography}
\end{document}

